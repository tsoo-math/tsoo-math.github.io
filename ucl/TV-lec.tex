% Options for packages loaded elsewhere
\PassOptionsToPackage{unicode}{hyperref}
\PassOptionsToPackage{hyphens}{url}
%
\documentclass[
]{article}
\usepackage{amsmath,amssymb}
\usepackage{lmodern}
\usepackage{ifxetex,ifluatex}
\ifnum 0\ifxetex 1\fi\ifluatex 1\fi=0 % if pdftex
  \usepackage[T1]{fontenc}
  \usepackage[utf8]{inputenc}
  \usepackage{textcomp} % provide euro and other symbols
\else % if luatex or xetex
  \usepackage{unicode-math}
  \defaultfontfeatures{Scale=MatchLowercase}
  \defaultfontfeatures[\rmfamily]{Ligatures=TeX,Scale=1}
\fi
% Use upquote if available, for straight quotes in verbatim environments
\IfFileExists{upquote.sty}{\usepackage{upquote}}{}
\IfFileExists{microtype.sty}{% use microtype if available
  \usepackage[]{microtype}
  \UseMicrotypeSet[protrusion]{basicmath} % disable protrusion for tt fonts
}{}
\makeatletter
\@ifundefined{KOMAClassName}{% if non-KOMA class
  \IfFileExists{parskip.sty}{%
    \usepackage{parskip}
  }{% else
    \setlength{\parindent}{0pt}
    \setlength{\parskip}{6pt plus 2pt minus 1pt}}
}{% if KOMA class
  \KOMAoptions{parskip=half}}
\makeatother
\usepackage{xcolor}
\IfFileExists{xurl.sty}{\usepackage{xurl}}{} % add URL line breaks if available
\IfFileExists{bookmark.sty}{\usepackage{bookmark}}{\usepackage{hyperref}}
\hypersetup{
  pdftitle={Coupling and total variational distance},
  hidelinks,
  pdfcreator={LaTeX via pandoc}}
\urlstyle{same} % disable monospaced font for URLs
\usepackage[margin=1in]{geometry}
\usepackage{color}
\usepackage{fancyvrb}
\newcommand{\VerbBar}{|}
\newcommand{\VERB}{\Verb[commandchars=\\\{\}]}
\DefineVerbatimEnvironment{Highlighting}{Verbatim}{commandchars=\\\{\}}
% Add ',fontsize=\small' for more characters per line
\usepackage{framed}
\definecolor{shadecolor}{RGB}{248,248,248}
\newenvironment{Shaded}{\begin{snugshade}}{\end{snugshade}}
\newcommand{\AlertTok}[1]{\textcolor[rgb]{0.94,0.16,0.16}{#1}}
\newcommand{\AnnotationTok}[1]{\textcolor[rgb]{0.56,0.35,0.01}{\textbf{\textit{#1}}}}
\newcommand{\AttributeTok}[1]{\textcolor[rgb]{0.77,0.63,0.00}{#1}}
\newcommand{\BaseNTok}[1]{\textcolor[rgb]{0.00,0.00,0.81}{#1}}
\newcommand{\BuiltInTok}[1]{#1}
\newcommand{\CharTok}[1]{\textcolor[rgb]{0.31,0.60,0.02}{#1}}
\newcommand{\CommentTok}[1]{\textcolor[rgb]{0.56,0.35,0.01}{\textit{#1}}}
\newcommand{\CommentVarTok}[1]{\textcolor[rgb]{0.56,0.35,0.01}{\textbf{\textit{#1}}}}
\newcommand{\ConstantTok}[1]{\textcolor[rgb]{0.00,0.00,0.00}{#1}}
\newcommand{\ControlFlowTok}[1]{\textcolor[rgb]{0.13,0.29,0.53}{\textbf{#1}}}
\newcommand{\DataTypeTok}[1]{\textcolor[rgb]{0.13,0.29,0.53}{#1}}
\newcommand{\DecValTok}[1]{\textcolor[rgb]{0.00,0.00,0.81}{#1}}
\newcommand{\DocumentationTok}[1]{\textcolor[rgb]{0.56,0.35,0.01}{\textbf{\textit{#1}}}}
\newcommand{\ErrorTok}[1]{\textcolor[rgb]{0.64,0.00,0.00}{\textbf{#1}}}
\newcommand{\ExtensionTok}[1]{#1}
\newcommand{\FloatTok}[1]{\textcolor[rgb]{0.00,0.00,0.81}{#1}}
\newcommand{\FunctionTok}[1]{\textcolor[rgb]{0.00,0.00,0.00}{#1}}
\newcommand{\ImportTok}[1]{#1}
\newcommand{\InformationTok}[1]{\textcolor[rgb]{0.56,0.35,0.01}{\textbf{\textit{#1}}}}
\newcommand{\KeywordTok}[1]{\textcolor[rgb]{0.13,0.29,0.53}{\textbf{#1}}}
\newcommand{\NormalTok}[1]{#1}
\newcommand{\OperatorTok}[1]{\textcolor[rgb]{0.81,0.36,0.00}{\textbf{#1}}}
\newcommand{\OtherTok}[1]{\textcolor[rgb]{0.56,0.35,0.01}{#1}}
\newcommand{\PreprocessorTok}[1]{\textcolor[rgb]{0.56,0.35,0.01}{\textit{#1}}}
\newcommand{\RegionMarkerTok}[1]{#1}
\newcommand{\SpecialCharTok}[1]{\textcolor[rgb]{0.00,0.00,0.00}{#1}}
\newcommand{\SpecialStringTok}[1]{\textcolor[rgb]{0.31,0.60,0.02}{#1}}
\newcommand{\StringTok}[1]{\textcolor[rgb]{0.31,0.60,0.02}{#1}}
\newcommand{\VariableTok}[1]{\textcolor[rgb]{0.00,0.00,0.00}{#1}}
\newcommand{\VerbatimStringTok}[1]{\textcolor[rgb]{0.31,0.60,0.02}{#1}}
\newcommand{\WarningTok}[1]{\textcolor[rgb]{0.56,0.35,0.01}{\textbf{\textit{#1}}}}
\usepackage{longtable,booktabs,array}
\usepackage{calc} % for calculating minipage widths
% Correct order of tables after \paragraph or \subparagraph
\usepackage{etoolbox}
\makeatletter
\patchcmd\longtable{\par}{\if@noskipsec\mbox{}\fi\par}{}{}
\makeatother
% Allow footnotes in longtable head/foot
\IfFileExists{footnotehyper.sty}{\usepackage{footnotehyper}}{\usepackage{footnote}}
\makesavenoteenv{longtable}
\usepackage{graphicx}
\makeatletter
\def\maxwidth{\ifdim\Gin@nat@width>\linewidth\linewidth\else\Gin@nat@width\fi}
\def\maxheight{\ifdim\Gin@nat@height>\textheight\textheight\else\Gin@nat@height\fi}
\makeatother
% Scale images if necessary, so that they will not overflow the page
% margins by default, and it is still possible to overwrite the defaults
% using explicit options in \includegraphics[width, height, ...]{}
\setkeys{Gin}{width=\maxwidth,height=\maxheight,keepaspectratio}
% Set default figure placement to htbp
\makeatletter
\def\fps@figure{htbp}
\makeatother
\setlength{\emergencystretch}{3em} % prevent overfull lines
\providecommand{\tightlist}{%
  \setlength{\itemsep}{0pt}\setlength{\parskip}{0pt}}
\setcounter{secnumdepth}{5}
\ifluatex
  \usepackage{selnolig}  % disable illegal ligatures
\fi

\title{Coupling and total variational distance}
\author{}
\date{\vspace{-2.5em}}

\usepackage{amsthm}
\newtheorem{theorem}{Theorem}[section]
\newtheorem{lemma}{Lemma}[section]
\newtheorem{corollary}{Corollary}[section]
\newtheorem{proposition}{Proposition}[section]
\newtheorem{conjecture}{Conjecture}[section]
\theoremstyle{definition}
\newtheorem{definition}{Definition}[section]
\theoremstyle{definition}
\newtheorem{example}{Example}[section]
\theoremstyle{definition}
\newtheorem{exercise}{Exercise}[section]
\theoremstyle{remark}
\newtheorem*{remark}{Remark}
\newtheorem*{solution}{Solution}
\begin{document}
\maketitle

{
\setcounter{tocdepth}{2}
\tableofcontents
}
\hypertarget{coupling}{%
\section{Coupling}\label{coupling}}

A \textbf{coupling} of two random variables \(X\) and \(Y\) is a pair of random variables \((X', Y')\) with a joint distribution such that its marginal distributions are given by those of \(X\) and \(Y\). Given two random variables \(X\) and \(Y\), we can't do simple operations on them such as \(X+Y\), unless we know their joint distribution, that is, if they even have one! If they have a joint distribution, then they live on the same probability space.

Often one can specify a coupling of \(X\) and \(Y\), if we can find a function \(F(X, U)\), where \(U\) is uniformly distributed on \([0,1]\) and independent of \(X\), so that \(F(X, U)\) has the same law as \(Y\). That is we can envision \(Y\) as a function of \(X\) and an additional randomization \(U\).

\hypertarget{examples}{%
\section{Examples}\label{examples}}

\begin{itemize}
\item
  Independent coupling: this option is always available, but it is often \emph{not} the most interesting or useful.
\item
  Quantile coupling/Inverse transform: If \(F_X\) and \(F_Y\) are cdfs, and \(U\) is uniformly distributed, then \(F_X^{-1}(U)\) and \(F_Y^{-1}(U)\) are random variables with these respective cdfs.
\item
  Thinning: If \(X\) is Bernoulli \(p\) and \(Z\) is an independent Bernoulli \(r\), then \(XZ\) is Bernoulli \(pr\). Thus \((X, XZ)\) is a coupling of Bernoulli random variables with parameters \(p\) and \(pr\), and \(X \geq XZ\).
\end{itemize}

\hypertarget{comparing-games}{%
\section{Comparing games}\label{comparing-games}}

Suppose June plays a gambling game with her mum using \(p=2/3\) coin, betting on heads, winning \(1\) pound if the coin comes up heads, and losing \(1\) pound otherwise. We start with \(0\) pounds and allow the possibility of going negative. June stops playing as soon as she reaches \(5\) pounds. Suppose Tessa plays the same game with her dad using a \(p=1/2\) coin.

We can't say that June will be better off than Tessa for \emph{sure}, since the coins are independent. Suppose we want to compare the expectation of the length of time \(W\) it takes for the game to end; it seems obvious that the mean time for June should be \emph{less} than that of Tessa's. In fact, the mean time for Tessa is infinite.

\hypertarget{enter-coupling}{%
\section{Enter coupling}\label{enter-coupling}}

We can envision a coupling of these two games, where June is \emph{always} better off.

\begin{itemize}
\item
  Let \(X_i\) be amount that June wins/loses on the \(i\)th flip, and \(Y_i\) is the corresponding amount for Tessa.
\item
  We can arrange a coupling of \(X_i\) and \(Y_i\) so that \(X_i' \geq Y_i'\), so that Tessa wins only when June wins. Use thinning!
\item
  Under this coupling, we do have that \(W'_J \leq W'_T\), so we can compute
  \[\mathbb{E} W_J = \mathbb{E} W'_J \leq \mathbb{E}  W'_T = \mathbb{E} W_T,\] without knowing too much about \(W\).
\end{itemize}

\hypertarget{some-r-code}{%
\section{Some R code}\label{some-r-code}}

Here, we provide some code to illustrates the two coupled games.

We first code and test the coupled coin-flips.

\begin{Shaded}
\begin{Highlighting}[]
\NormalTok{coupled}\OtherTok{\textless{}{-}}\ControlFlowTok{function}\NormalTok{()\{}
\NormalTok{  j}\OtherTok{=} \FunctionTok{rbinom}\NormalTok{(}\DecValTok{1}\NormalTok{,}\DecValTok{1}\NormalTok{,}\DecValTok{2}\SpecialCharTok{/}\DecValTok{3}\NormalTok{)}
\NormalTok{  t}\OtherTok{=} \FunctionTok{rbinom}\NormalTok{(}\DecValTok{1}\NormalTok{,}\DecValTok{1}\NormalTok{,}\DecValTok{3}\SpecialCharTok{/}\DecValTok{4}\NormalTok{)}\SpecialCharTok{*}\NormalTok{j}
\FunctionTok{c}\NormalTok{(j,t)}
\NormalTok{\}}
\NormalTok{S}\OtherTok{=}\FunctionTok{replicate}\NormalTok{(}\DecValTok{10000}\NormalTok{, }\FunctionTok{coupled}\NormalTok{())}
\FunctionTok{mean}\NormalTok{(S[}\DecValTok{1}\NormalTok{,])}
\end{Highlighting}
\end{Shaded}

\begin{verbatim}
## [1] 0.6727
\end{verbatim}

\begin{Shaded}
\begin{Highlighting}[]
\FunctionTok{mean}\NormalTok{(S[}\DecValTok{2}\NormalTok{,])}
\end{Highlighting}
\end{Shaded}

\begin{verbatim}
## [1] 0.5063
\end{verbatim}

Then, we run simulations to see long the game takes on average for June and Tessa.

\begin{Shaded}
\begin{Highlighting}[]
\NormalTok{exit}\OtherTok{\textless{}{-}}\ControlFlowTok{function}\NormalTok{(p)\{}
\NormalTok{  x}\OtherTok{=}\DecValTok{0}
\NormalTok{  n}\OtherTok{=}\DecValTok{0}
  \ControlFlowTok{while}\NormalTok{(x }\SpecialCharTok{\textless{}}\DecValTok{5}\NormalTok{)\{}
\NormalTok{    x }\OtherTok{\textless{}{-}}\NormalTok{ x}\SpecialCharTok{+}\NormalTok{(}\DecValTok{2}\SpecialCharTok{*}\FunctionTok{rbinom}\NormalTok{(}\DecValTok{1}\NormalTok{,}\DecValTok{1}\NormalTok{,p) }\SpecialCharTok{{-}}\DecValTok{1}\NormalTok{)}
\NormalTok{  n }\OtherTok{\textless{}{-}}\NormalTok{ n}\SpecialCharTok{+}\DecValTok{1}
\NormalTok{  \}}
\NormalTok{  n}
\NormalTok{\}}

\FunctionTok{mean}\NormalTok{(}\FunctionTok{replicate}\NormalTok{(}\DecValTok{1000}\NormalTok{, }\FunctionTok{exit}\NormalTok{(}\DecValTok{2}\SpecialCharTok{/}\DecValTok{3}\NormalTok{)))}
\end{Highlighting}
\end{Shaded}

\begin{verbatim}
## [1] 14.476
\end{verbatim}

\begin{Shaded}
\begin{Highlighting}[]
\FunctionTok{mean}\NormalTok{(}\FunctionTok{replicate}\NormalTok{(}\DecValTok{25}\NormalTok{, }\FunctionTok{exit}\NormalTok{(}\DecValTok{1}\SpecialCharTok{/}\DecValTok{2}\NormalTok{)))}
\end{Highlighting}
\end{Shaded}

\begin{verbatim}
## [1] 1838.2
\end{verbatim}

We could of run similar simulations without coupling. The point is, we established this result analytically using coupling. The code here is really just to illustrate the coupling. In the last bit of code, we note that in each simulation, under the coupling, Tessa never finishes before June.

\begin{Shaded}
\begin{Highlighting}[]
\NormalTok{exitcoupled}\OtherTok{\textless{}{-}}\ControlFlowTok{function}\NormalTok{()\{}
\NormalTok{  june }\OtherTok{=}\DecValTok{0}
\NormalTok{  tessa }\OtherTok{=}\DecValTok{0}
\NormalTok{  njune}\OtherTok{=}\DecValTok{0}
\NormalTok{  ntessa}\OtherTok{=}\DecValTok{0}
  \ControlFlowTok{while}\NormalTok{(june }\SpecialCharTok{\textless{}} \DecValTok{5}\NormalTok{)\{}
\NormalTok{    v }\OtherTok{=} \FunctionTok{coupled}\NormalTok{()}
\NormalTok{    june }\OtherTok{\textless{}{-}}\NormalTok{ june }\SpecialCharTok{+} \DecValTok{2}\SpecialCharTok{*}\NormalTok{v[}\DecValTok{1}\NormalTok{] }\SpecialCharTok{{-}}\DecValTok{1}
\NormalTok{    tessa }\OtherTok{\textless{}{-}}\NormalTok{ tessa }\SpecialCharTok{+} \DecValTok{2}\SpecialCharTok{*}\NormalTok{v[}\DecValTok{2}\NormalTok{] }\SpecialCharTok{{-}}\DecValTok{1}
\NormalTok{    njune }\OtherTok{\textless{}{-}} \DecValTok{1}\SpecialCharTok{+}\NormalTok{njune}
\NormalTok{    ntessa }\OtherTok{\textless{}{-}} \DecValTok{1} \SpecialCharTok{+}\NormalTok{ ntessa\}}
  \ControlFlowTok{while}\NormalTok{(tessa }\SpecialCharTok{\textless{}}\DecValTok{5}\NormalTok{)\{}
\NormalTok{    tessa }\OtherTok{\textless{}{-}}\NormalTok{ tessa}\SpecialCharTok{+} \DecValTok{2}\SpecialCharTok{*}\FunctionTok{rbinom}\NormalTok{(}\DecValTok{1}\NormalTok{,}\DecValTok{1}\NormalTok{,}\FloatTok{0.5}\NormalTok{) }\SpecialCharTok{{-}}\DecValTok{1}
\NormalTok{ntessa }\OtherTok{\textless{}{-}} \DecValTok{1} \SpecialCharTok{+}\NormalTok{ ntessa}
\NormalTok{  \}}
\FunctionTok{c}\NormalTok{(njune, ntessa)  }
\NormalTok{\}}
\end{Highlighting}
\end{Shaded}

\begin{Shaded}
\begin{Highlighting}[]
\FunctionTok{replicate}\NormalTok{(}\DecValTok{25}\NormalTok{, }\FunctionTok{exitcoupled}\NormalTok{())}
\end{Highlighting}
\end{Shaded}

\begin{verbatim}
##      [,1] [,2] [,3] [,4] [,5] [,6] [,7] [,8] [,9]
## [1,]    9   13    7   15   15    9   19    9    5
## [2,] 1491   95    9  503   77    9 4035   67   67
##      [,10] [,11] [,12] [,13] [,14] [,15] [,16]
## [1,]     9    29     7    11    13    31    17
## [2,]    43   125     7    25    15    81   119
##      [,17] [,18] [,19] [,20] [,21] [,22] [,23]
## [1,]    15    43     7     7    17     5    19
## [2,]    29  1055    37   107 30317    37    59
##      [,24] [,25]
## [1,]    19    15
## [2,]    21  2491
\end{verbatim}

\hypertarget{total-variational-distance}{%
\section{Total variational distance}\label{total-variational-distance}}

Let \(X\) and \(Y\) be real-valued random variables. The \textbf{total variational} distance between \(X\) and \(Y\) is given by:\\
\[d_{TV}(X,Y) = 2 \cdot \sup_{A  \text{ is an event}} | \mathbb{P}(X \in A) - \mathbb{P}(Y \in A)|.\] Note that the definition of \(d_{TV}(X,Y)\) only depends on the laws of \(X\) and \(Y\), individually; in particular, if we have that if \((X',Y')\) is a coupling of \(X\) and \(Y\), then \[d_{TV}(X',Y') = d_{TV}(X,Y).\]

\hypertarget{common-formula}{%
\subsection{Common formula}\label{common-formula}}

\begin{lemma}
\protect\hypertarget{lem:unnamed-chunk-5}{}{\label{lem:unnamed-chunk-5} } Let \(X\) and \(Y\) be integer-valued random variables. We have that
\[d_{TV}(X,Y) = \sum_{z \in \mathbb{Z}} | \mathbb{P}(X=z) - \mathbb{P}(Y=z)|.\]\\
\end{lemma}

The proof follows from the following simple fact:
If \[D = \{z \in \mathbb{Z}:  \mathbb{P}(X =z) \geq \mathbb{P}(Y=z)\},\] then

\begin{eqnarray*} \sum_{z \in \mathbb{Z}} | \mathbb{P}(X=z) - \mathbb{P}(Y=z)| 
&=& \sum_{z \in D} (\mathbb{P}(X=z) - \mathbb{P}(Y=z))+\sum_{z \in D^c} (\mathbb{P}(Y=z) - \mathbb{P}(X=z)).
\end{eqnarray*}

\begin{proof}
\iffalse{} {Proof. } \fi{}Observe that for any \(A \subset \mathbb{Z}\), we have
\[|\mathbb{P}(X \in A) - \mathbb{P}(Y \in A)| = |\mathbb{P}(X \in A^c) - \mathbb{P}(Y\in A^c)|.\]
Thus using the set \(D\), we see that
\[d_{TV}(X,Y) \geq \sum_{z \in \mathbb{Z}} | \mathbb{P}(X=z) - \mathbb{P}(Y=z)|.\]

For the other direction, note that

\begin{eqnarray*}
        |\mathbb{P}(X \in A) - \mathbb{P}(Y \in A)| &=& \big|\sum_{z \in A} \mathbb{P}(X =z) - \sum_{z \in A} \mathbb{P}(Y=z) \big| \\
        &\leq& \sum_{z \in A} |\mathbb{P}(X=z) - \mathbb{P}(Y=z)|.
\end{eqnarray*}

Thus it follows that

\begin{eqnarray*}
    2| \mathbb{P}(X \in A) - \mathbb{P}(Y \in A)| 
&=& |\mathbb{P}(X \in A) - \mathbb{P}(Y \in A)| + |\mathbb{P}(X \in A^c)  - \mathbb{P}(Y \in A^c)| \\
        &\leq&   \sum_{z \in \mathbb{Z}} |\mathbb{P}(X=z) - \mathbb{P}(Y=z)|.
    \end{eqnarray*}
\end{proof}

\begin{exercise}
\protect\hypertarget{exr:unnamed-chunk-7}{}{\label{exr:unnamed-chunk-7} }Let \(X \sim Bern(p)\), \(Y \sim Bern(q)\), and \(W \sim Poi(p)\). Compute
\(d_{TV}(X,Y)\) and \(d_{TV}(X, W).\)\\
\end{exercise}

\hypertarget{coupling-inequality}{%
\subsection{Coupling inequality}\label{coupling-inequality}}

\begin{lemma}
\protect\hypertarget{lem:unnamed-chunk-8}{}{\label{lem:unnamed-chunk-8} }If \(X\) and \(Y\) are jointly distributed random variables, then

\[d_{TV}(X,Y) \leq 2\mathbb{P}(X \not = Y).\]
\end{lemma}

\begin{proof}
\iffalse{} {Proof. } \fi{}Let \(A\) be an event. Note that

\begin{eqnarray*}
        |\mathbb{P}(X \in A) - \mathbb{P}(Y \in A)|
    &=& \Big| \mathbb{P}(X \in A, X = Y) +  \mathbb{P}(X \in A, X\not = Y) -   \mathbb{P}(Y \in A, X\not = Y)- \mathbb{P}(Y \in A, X = Y) \Big| \\
        &=&   |\mathbb{P}(X \in A, X \not = Y) -  \mathbb{P}(Y \in A, X\not = Y) | \\
        &\leq&
        \mathbb{P}(X \not = Y).
    \end{eqnarray*}
\end{proof}

\hypertarget{examples-1}{%
\section{Examples}\label{examples-1}}

\begin{itemize}
\tightlist
\item
  Let \(X_1, \ldots, X_n, X_{n+1}\) be i.i.d. Bernoulli with parameter \(p\). Let \(S_k\) be their partial sum. We can bound
\end{itemize}

\begin{eqnarray*}
d_{TV}(S_n, S_{n+1}) &\leq& 2\mathbb{P}(S_n \not = S_{n+1}) \\
&=&  2\mathbb{P}(X_{n+1} = 1) \\
&=&  2p
\end{eqnarray*}

\begin{itemize}
\tightlist
\item
  Let \(X\) be a six-sided fair die, and \(Y\) be a seven-sided fair die. Assume \(X\) and \(Y\) are independent. Let \(X' = Y\), if \(Y \not = 7\), otherwise, take \(X' = X\). Clearly, \(X'\) has the same law as a six-sided fair die. Then
  \[d_{TV}(X, Y)  = d_{TV}(X', Y) \leq 2\mathbb{P}(X' \not= Y )= \frac{2}{7}.\]
\end{itemize}

\hypertarget{summary}{%
\section{Summary}\label{summary}}

We saw some basics of how coupling can be used as a tool in probability theory and in particular, how it can be used the bound the total variational distance between two random variables.

\hypertarget{version-28-april-2021}{%
\section{Version: 28 April 2021}\label{version-28-april-2021}}

\begin{itemize}
\tightlist
\item
  \href{https://tsoo-math.github.io/ucl/TV-lec.Rmd}{Rmd Source}
\end{itemize}

\end{document}
